% ==========================================
% BAB VI EVALUASI
% ==========================================
\cleardoublepage
\chapter{EVALUASI}
\label{chap:evaluasi}
asdfasdf
\section{Metode Evaluasi}
asdfasdf
\section{Hasil Evaluasi}
asdfasdf
\section{Pembahasan Hasil Evaluasi}
\section{Evaluasi Kesalahan Penulisan yang Sering Terjadi}
\subsection{Penggunaan Kata "di mana" atau "dimana"}
Banyak yang menuliskan kata "di mana" atau "dimana" sebagai pengganti kata "which" dalam bahasa Inggris. 
Padahal, penggunaan kata "di mana" atau "dimana" tidak tepat dalam konteks tersebut. Demikian juga untuk kata serupa, misalnya "yang mana".
Kata "di mana" atau "dimana" ini harus diganti dengan kata lain, seperti "dengan", "tempat", "yang", dan sebagainya tergantung kalimatnya.
Penjelasan lengkap dapat dilihat pada \autocite{BPBI}.

\subsection{Penggunaan Kata "sedangkan" dan "sehingga"}

\begin{table}[t]
  \begin{tabular}{|c|l|l|}
  \hline
  Kata & Salah & Benar \\ \hline
  sedangkan & \begin{tabular}[c]{@{}c@{}}Sedangkan sistem lama masih\\ digunakan oleh banyak pengguna.\end{tabular} & \begin{tabular}[c]{@{}c@{}}Sistem lama masih digunakan\\ oleh banyak pengguna,\\ sedangkan sistem baru belum siap.\end{tabular} \\ \hline
  sehingga & \begin{tabular}[c]{@{}c@{}}Sehingga sistem lama masih\\ digunakan oleh banyak pengguna.\end{tabular} & \begin{tabular}[c]{@{}c@{}}Sistem lama masih digunakan\\ oleh banyak pengguna sehingga\\ sistem baru belum siap.\end{tabular} \\ \hline
  \end{tabular}
  \caption{Contoh penggunaan kata "sedangkan" dan "sehingga"}
  \label{tbl:sedangkan_sehingga}
\end{table}

Kata "sedangkan" dan "sehingga" adalah kata hubung atau konjungsi. 
Konjungsi adalah kata atau ungkapan yang menghubungkan satuan bahasa 
(kata, frasa, klausa, dan kalimat). 
Konjungsi dapat dibagi menjadi konjungsi intrakalimat dan antarkalimat.  
Kata "sedangkan" menghubungkan dua klausa yang bersifat kontrasif, 
sedangkan "sehingga" menghubungkan dua klausa yang bersifat kausal. 
Dalam ragam formal, kata hubung “sedangkan” dan “sehingga” hanya dapat digunakan 
sebagai konjungsi intrakalimat sehingga kedua konjungsi itu \textbf{tidak dapat diletakkan pada awal kalimat}.
Selain itu, penggunaan kata "sedangkan" harus didahului oleh koma (,), sedangkan kata "sehingga" tidak perlu didahului oleh koma (,).
Contoh penggunaan yang benar dan salah dapat dilihat pada Tabel \ref{tbl:sedangkan_sehingga}.


\subsection{Penggunaan Istilah yang Tidak Baku}
Ada beberapa istilah yang sering digunakan dalam pembicaraan sehari-hari, tetapi tidak baku dalam penulisan ilmiah.
Beberapa istilah tersebut antara lain:
\begin{enumerate}
  \item analisa $\rightarrow$ analisis
  \item eksisting atau existing $\rightarrow$ yang ada atau saat ini
  \item bisnis proses $\rightarrow$ proses bisnis
  \item user $\rightarrow$ pengguna
  \item system $\rightarrow$ sistem
  \item database $\rightarrow$ basis data
  \item aktifitas $\rightarrow$ aktivitas
  \item efektifitas $\rightarrow$ efektivitas
  \item sosial media $\rightarrow$ media sosial
\end{enumerate}

\subsection{Pemisah Desimal dan Ribuan}
Tanda pemisah desimal dalam bahasa Indonesia adalah tanda koma, contoh:
\begin{enumerate}
  \item (Salah) Akurasi naik menjadi 50.6\% 
  \item (Benar) Akurasi naik menjadi 50,6\% 
\end{enumerate}

\subsection{Daftar atau \textit{List}}
Ada beberapa aturan penulisan daftar atau \textit{list} yang perlu diperhatikan, antara lain:
\begin{enumerate}[a)]
\item Jika memungkinkan, hindari penggunaan “bullet points” atau sejenisnya. Sebaiknya, gunakan angka (1, 2, 3, ...) atau huruf (a, b, c, …). Dengan demikian, pembaca dapat dengan mudah melihat jumlah \textit{item} atau \textit{list}. 
\item Jika dalam daftar hanya ada satu item, tidak perlu menggunakan nomor urut.
\item Penjelasan atau deskripsi suatu item sebaiknya menyatu dengan judul item tersebut, tidak berbeda halaman. Contoh yang salah: judul item ada di halaman 10, namun deskripsinya di halaman 11. Sebaiknya pindahkan judul tersebut ke halaman 11.
\item Jika penjelasan atau deskripsi suatu item cukup panjang, misalnya lebih dari 1 halaman atau terdiri atas beberapa paragraf, sebaiknya setiap item tersebut dijadikan judul subbab, kecuali jika level subbab sudah mencapai level 4. 
\end{enumerate}

\subsection{Penggunaan Kata "masing-masing" dan "setiap"}
Kata “masing-masing” digunakan di belakang kata yang diterangkan, misalnya 
"Setiap proses menggunakan algoritma masing-masing". Kata “tiap-tiap” atau “setiap”
ditempatkan di depan kata yang diterangkan, misalnya
"Setiap proses menggunakan algoritma tertentu".