\chapter*{ABSTRAK}
\addcontentsline{toc}{chapter}{ABSTRAK}

\begin{center}

    \textbf{Implementasi Sistem Smart Scale Berbasis IoT dengan Arsitektur Desentralisasi untuk Optimalisasi Rantai Pasok Jeruk di Tingkat Lapak} \\
    \vspace{0.4cm}
    oleh \\
    \vspace{0.4cm}
    John Doe\\
    123456789\\
\end{center}


\begin{spacing}{1.0}
Proses manual dalam rantai pasok jeruk di tingkat lapak, yang mencakup sortir,
timbang, dan catat, terbukti tidak efisien, memakan waktu, dan rentan terhadap
kesalahan. Meskipun solusi berbasis Internet of Things (IoT) dapat mengatasi
masalah ini, arsitektur terpusat konvensional berbasis cloud justru menimbulkan
masalah baru berupa latensi tinggi yang menghambat alur kerja real-time.
Penelitian ini bertujuan untuk merancang dan mengimplementasikan sebuah
purwarupa sistem smart scale berbasis IoT dengan pendekatan desentralisasi (edge
computing) untuk meminimalkan latensi dan meningkatkan efisiensi operasional.
Metode yang digunakan adalah Model Purwarupa, yang diwujudkan dalam bentuk
perangkat keras dengan arsitektur prosesor ganda (Raspberry Pi 5 dan ESP32) yang
mampu mengintegrasikan fungsi timbang dan analisis kualitas citra secara
simultan. Hasil evaluasi menunjukkan bahwa purwarupa dengan arsitektur edge
computing yang diimplementasikan mampu memberikan respons 43,5\% lebih cepat
(latensi rata-rata 18,05 detik) dibandingkan dengan simulasi arsitektur terpusat.
Selain itu, sistem ini berhasil meningkatkan efisiensi waktu kerja secara
keseluruhan sebesar 58,32\% jika dibandingkan dengan metode manual
konvensional. Pengujian Penerimaan Pengguna (UAT) yang melibatkan 30
partisipan juga menunjukkan penerimaan yang sangat positif, dengan skor rata-rata
untuk kemudahan penggunaan (5,0/5,0), kejelasan antarmuka (4,87/5,0) dan
persepsi kinerja (4,95/5,0). Kesimpulannya, implementasi sistem smart scale
dengan arsitektur desentralisasi terbukti merupakan solusi yang valid dan efektif
untuk menjawab permasalahan latensi dan inefisiensi di lapak. Sistem ini tidak
hanya unggul secara teknis dalam hal performa, tetapi juga fungsional dan dapat
diterima dengan baik oleh pengguna akhir, serta berpotensi besar untuk modernisasi
rantai pasok agribisnis.

Kata kunci: Smart Scale, Internet of Things, Edge Computing, Rantai Pasok Jeruk.
\end{spacing}

