% ============================================================================================
% BAB III ANALISIS MASALAH
% Pembagian subbab tidak rigid dan dapat bervariasi. Bab ini minimal berisi analisis kebutuhan
% fungsional dan nonfungsional, analisis berbagai alternatif solusi yang dapat ditawarkan, dan
% metode pemilihan solusi yang diusulkan.
% ============================================================================================
\cleardoublepage
\chapter{ANALISIS MASALAH}
\label{chap:analisis-masalah}
Bab ini membahas analisis masalah yang ada pada sistem saat ini, analisis kebutuhan yang diperlukan
untuk menyelesaikan masalah tersebut, serta analisis pemilihan solusi yang diusulkan. Secara garis besar,
bab ini membahas hal-hal berikut:
\begin{enumerate}[1.]
    \item Deskripsi hasil analisis dari persoalan yang sedang ditangani dalam TA;
    \item Penjelasan tentang solusi yang ditawarkan (dapat berupa algoritma, konsep desain, formula, atau gabungannya) 
    pada level konsep dan detail yang cukup untuk merealisasikan solusinya; dan
    \item Penjelasan tentang pendekatan yang dipilih dalam menyelesaikan persoalan TA.
\end{enumerate}
Subbab-subbab dalam bab ini dapat disesuaikan dengan kebutuhan topik TA yang dibahas. Berikut adalah contoh pembagian subbab
yang dapat digunakan sebagai referensi (judul subbab dapat diubah sesuai dengan hal yang hendak dibahas dalam subbab tersebut).

\section{Analisis Kondisi Saat Ini}
Menurut \textcite{laudon2020}, gambarkan terlebih dahulu model konseptual sistem yang ada saat ini. Model konseptual ini berisi berbagai komponen atau subsitem dan interaksi antarsubsistem tersebut. Setelah itu, berikan penjelasan tentang masalah yang ada pada sistem tersebut. Paragraf berikut berisi contoh penjabaran masalah sistem informasi fasilitas kesehatan untuk pasien \autocite{pressman2019}. 
\section{Analisis Kebutuhan}
\lipsum[4]
\subsection{Identifikasi Masalah Pengguna}
\lipsum[5]
\subsection{Kebutuhan Fungsional}
\lipsum[6]
\subsection{Kebutuhan Nonfungsional}
\lipsum[7]

\section{Analisis Pemilihan Solusi}
\subsection{Alternatif Solusi}
\lipsum[8]
\subsection{Analisis Penentuan Solusi}
\lipsum[9]